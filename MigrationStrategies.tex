% $Author: oscar $
% $Date: 2009-09-15 16:53:48 +0200 (Tue, 15 Sep 2009) $
% $Revision: 29111 $
%=================================================================
\ifx\wholebook\relax\else
% --------------------------------------------
% Lulu:
	\documentclass[a4paper,10pt,twoside]{book}
	\usepackage[
		papersize={6.13in,9.21in},
		hmargin={.815in,.815in},
		vmargin={.98in,.98in},
		ignoreheadfoot
	]{geometry}
	\usepackage[hangul]{kotex}
	% $Author: oscar $
% $Date: 2009-09-13 20:58:29 +0200 (Sun, 13 Sep 2009) $
% $Revision: 29070 $
%=============================================================
% NB: documentclass must be set in main document.
% Allows book to be generated in multiple formats.
%=============================================================
%:Packages
\usepackage[T1]{fontenc}  %%%%%really important to get the code directly in the text!
\usepackage{palatino}
\usepackage{ifthen}
\usepackage{graphicx}
\graphicspath{{figures/}}
\usepackage{xspace}
\usepackage{makeidx}
\usepackage{isodateo} % enable \isodate
\usepackage{amssymb,textcomp}
%=============================================================
%:More packages
%\usepackage[english]{babel}
%\usepackage{lmodern}
%\usepackage[scaled=0.85]{helvet}
%\usepackage{microtype}
%\usepackage{theorem}
%\usepackage{float}
%\usepackage{longtable}
%\usepackage[nottoc]{tocbibind}
%\usepackage{multicol}
%\usepackage{booktabs}	% book-style tables
%\usepackage{topcapt}	% enables \topcaption
%\usepackage{multirow}
%\usepackage{tabularx}
%\usepackage{alltt}
\usepackage[usenames,dvipsnames]{color}
%\usepackage[hang]{subfigure}\makeatletter\def\p@subfigure{\thefigure\,}\makeatother
%\usepackage{rotating}
%\usepackage{enumitem}	% apb: allows more control over tags in enumerations
%\usepackage{verbatim}     % for comment environment
%\usepackage{varioref}	% for page references that work
%\usepackage{needspace}
%\usepackage[newparttoc]{titlesec}
%\usepackage{titletoc}
%\usepackage{wrapfig}
\usepackage[
	colorlinks=true,
	linkcolor=black,
	urlcolor=black,
	citecolor=black
]{hyperref}   % should come last
%=============================================================
%:URL style
\makeatletter
\def\url@leostyle{%
  \@ifundefined{selectfont}{\def\UrlFont{\sf}}{\def\UrlFont{\sffamily}}}
\makeatother
\urlstyle{leo}
%=============================================================
%:Booleans
\newboolean{lulu}
\setboolean{lulu}{false}
\newcommand{\ifluluelse}[2]{\ifthenelse{\boolean{lulu}}{#1}{#2}}
%=============================================================
%:Editorial comment macros
\newcommand{\nnbb}[2]{
  \fbox{\bfseries\sffamily\scriptsize#1}
  {\sf\small$\blacktriangleright$\textit{#2}$\blacktriangleleft$}
}
\newcommand{\on}[1]{\nnbb{Oscar}{#1}}
\newcommand{\here}{\nnbb{CONTINUE}{HERE}}
%=============================================================
%:Abbreviation macros
\newcommand{\ie}{\emph{i.e.},\xspace}
\newcommand{\eg}{\emph{e.g.},\xspace}
\newcommand{\etc}{\emph{etc.}\xspace}
\newcommand{\etal}{\emph{et al.}\xspace}
\newcommand{\straightquote}{"}
\newcommand{\sba}{\url{SquareBracketAssociates.org}\xspace}
%=============================================================
%:Patterns
% \newcommand{\pattern}[2]{\newpage\section{{\sf #1}}\label{pat:#2}}
% \newcommand{\pattern}[2]{\newpage\index{#1 (Pattern)}\section{#1}\label{pat:#2}}
\newcommand{\pattern}[2]{\cleardoublepage\index{#1 (패턴)}\section{#1}\label{pat:#2}}
\newcommand{\thumbnail}[2]{\index{#1 (패턴)}\subsection{#1}\label{pat:#2}}
\newcommand{\thumblang}[2]{\index{#1 (패턴 랭귀지)}\subsection{#1}\label{pat:#2}}
\newcommand{\variant}[1]{{\emph{#1}}\xspace}
% \newcommand{\problem}[1]{\subsection*{Problem}\emph{#1}}
\newcommand{\intent}[1]{\paragraph{의도}\emph{#1}}
\newcommand{\problem}[1]{\paragraph{문제}\emph{#1}}
\newcommand{\solution}[1]{\paragraph{해결}\emph{#1}}
\newcommand{\discussion}[0]{\paragraph{토론}}
\newcommand{\cmd}[1]{{\tt #1}\xspace}
%=============================================================
%:Environments
\newenvironment{bulletlist}{\begin{itemize}\setlength{\itemsep}{0ex}}
{\end{itemize}}
%=============================================================
%:Cross reference macros
\newcommand{\chalabel}[1]{\label{cha:#1}}
\newcommand{\seclabel}[1]{\label{sec:#1}}
\newcommand{\figlabel}[1]{\label{fig:#1}}
\newcommand{\tablabel}[1]{\label{tab:#1}}
\newcommand{\rulelabel}[1]{\label{rule:#1}}
\newcommand{\eglabel}[1]{\label{eg:#1}}
\newcommand{\scrlabel}[1]{\label{scr:#1}}
\newcommand{\mthlabel}[1]{\label{mth:#1}}
\newcommand{\clslabel}[1]{\label{cls:#1}}
\newcommand{\faqlabel}[1]{\label{faq:#1}}
%\newcommand{\charef}[1]{Chapter~\ref{cha:#1}\xspace}
%\newcommand{\secref}[1]{Section~\ref{sec:#1}\xspace}
\newcommand{\figref}[1]{Figure~\ref{fig:#1}\xspace}
% \newcommand{\patpgref}[2]{\hyperref[pat:#2]{\sf #1} [p.~\pageref{pat:#2}]\xspace}
\newcommand{\patpgref}[2]{\index{#1 (Pattern)}\hyperref[pat:#2]{#1} [p.~\pageref{pat:#2}]\xspace}
\newcommand{\patlangpgref}[2]{\index{#1 (Pattern language)}\hyperref[pat:#2]{#1} [p.~\pageref{pat:#2}]\xspace}
% \newcommand{\patref}[2]{\hyperref[pat:#2]{\sf #1}\xspace}
\newcommand{\patref}[2]{\index{#1 (Pattern)}\hyperref[pat:#2]{#1}\xspace}
\newcommand{\patlangref}[2]{\index{#1 (Pattern language)}\hyperref[pat:#2]{#1}\xspace}
% \newcommand{\charef}[2]{\hyperref[cha:#2]{\underline{\sf #1}}\xspace}
% \newcommand{\charef}[2]{\hyperref[cha:#2]{\sf #1}\xspace}
\newcommand{\charef}[2]{\index{#1 (Pattern cluster)}\hyperref[cha:#2]{#1}\xspace}
% \newcommand{\chapgref}[2]{\hyperref[cha:#2]{\sf #1} [p.~\pageref{cha:#2}]\xspace}
\newcommand{\chapgref}[2]{\index{#1 (Pattern cluster)}\hyperref[cha:#2]{#1} [p.~\pageref{cha:#2}]\xspace}
%\newcommand{\Figref}[1]{Figure~\ref{fig:#1}\xspace}
%\newcommand{\appref}[1]{Appendix~\ref{app:#1}\xspace}
%\newcommand{\tabref}[1]{Table~\ref{tab:#1}\xspace}
%\newcommand{\ruleref}[1]{\ref{rule:#1}\xspace}
%\newcommand{\egref}[1]{example~\ref{eg:#1}\xspace}
%\newcommand{\Egref}[1]{Example~\ref{eg:#1}\xspace}
%\newcommand{\scrref}[1]{script~\ref{scr:#1}\xspace}
%\newcommand{\Scrref}[1]{Script~\ref{scr:#1}\xspace}
%\newcommand{\tscrref}[1]{the script~\ref{scr:#1}\xspace}
%\newcommand{\Tscrref}[1]{The script~\ref{scr:#1}\xspace}
%\newcommand{\mthref}[1]{method~\ref{mth:#1}\xspace}
%\newcommand{\mthsref}[1]{methods~\ref{mth:#1}\xspace}
%\newcommand{\Mthref}[1]{Method~\ref{mth:#1}\xspace}
%\newcommand{\tmthref}[1]{the method~\ref{mth:#1}\xspace}
%\newcommand{\Tmthref}[1]{The method~\ref{mth:#1}\xspace}
%\newcommand{\clsref}[1]{class~\ref{cls:#1}\xspace}
%\newcommand{\tclsref}[1]{the class~\ref{cls:#1}\xspace}
%\newcommand{\Tclsref}[1]{The class~\ref{cls:#1}\xspace}
%=============================================================
%:Page Layout
\setlength{\headsep}{1cm}
%=============================================================
%:Menu item macro
%\definecolor{lightgray}{gray}{0.89}
%\newcommand{\menu}[1]{{%
%	\setlength{\fboxsep}{0pt}%
%	\colorbox{lightgray}{{{\upshape\sffamily\strut \,#1\,}}}}}
%\newcommand{\go}{\,$\triangleright$\,}
%\newcommand{\short}[1]{\mbox{{\sc cmd}\hspace{0.08em}--\hspace{0.09em}#1}\xspace}
%\newcommand{\button}[1]{{%
%	\setlength{\fboxsep}{0pt}%
%	\fbox{{\upshape\sffamily\strut \,#1\,}}}}
%\newcommand{\toolsflap}{\textit{Tools} flap\xspace}
%=============================================================
%:Section depth
%\setcounter{secnumdepth}{2}
%
%\DeclareGraphicsExtensions{.pdf, .jpg, .png}
%=============================================================
%:PDF setup
\hypersetup{
   pdftitle={Object-Oriented Reengineering Patterns},
   pdfauthor={Serge Demeyer, St\'ephane Ducasse, Oscar Nierstrasz},
   pdfkeywords={Reengineering, Object-Oriented Programming, Patterns},
   pdfsubject={Computer Science}
}
%=============================================================
%:Page layout and appearance
%\renewcommand{\chaptermark}[1]{\markboth{#1}{}}
%\renewcommand{\sectionmark}[1]{\markright{\thesection\ #1}}
%\renewpagestyle{plain}[\small\itshape]{%
%	\setheadrule{0pt}%
%	\sethead[][][]{}{}{}%
%	\setfoot[][][]{}{}{}}
%\renewpagestyle{headings}[\small\itshape]{%
%	\setheadrule{0pt}%
%	\setmarks{chapter}{section}%
%	\sethead[\thepage][][\chaptertitle]{\sectiontitle}{}{\thepage}%
%	\setfoot[][][]{}{}{}}
%=============================================================
%:Title section setup and TOC numbering depth
%\setcounter{secnumdepth}{1}
%\setcounter{tocdepth}{1}
%\titleformat{\part}[display]{\centering}{\huge\partname\ \thepart}{1em}{\Huge\textbf}[]
%\titleformat{\chapter}[display]{}{\huge\chaptertitlename\ \thechapter}{1em}{\Huge\raggedright\textbf}[]
%\titlecontents{part}[3pc]{%
%		\pagebreak[2]\addvspace{1em plus.4em minus.2em}%
%		\leavevmode\large\bfseries}
%	{\contentslabel{3pc}}{\hspace*{-3pc}}
%	{}[\nopagebreak]
%\titlecontents{chapter}[3pc]{%
%		\pagebreak[0]\addvspace{1em plus.2em minus.2em}%
%		\leavevmode\bfseries}
%	{\contentslabel{3pc}}{}
%	{\hfill\contentspage}[\nopagebreak]
%\dottedcontents{section}[3pc]{}{3pc}{1pc}
%\dottedcontents{subsection}[3pc]{}{0pc}{1pc}
%\let\origdoublepage\cleardoublepage
%\newcommand{\clearemptydoublepage}{%
%  \clearpage
%  {\pagestyle{empty}\origdoublepage}}
%\let\cleardoublepage\clearemptydoublepage % see http://www.tex.ac.uk/cgi-bin/texfaq2html?label=patch
%=============================================================
%:Listings package configuration
\newcommand{\caret}{\makebox{\raisebox{0.4ex}{\footnotesize{$\wedge$}}}}
% \newcommand{\escape}{{\sf \textbackslash}}
\definecolor{source}{gray}{0.95}
\usepackage{listings}
\lstdefinelanguage{Smalltalk}{
  morestring=[d]',
% Adapt this to other languages!
%  morecomment=[s]{"}{"},
  alsoletter={\#:},
  %escapechar={!},
  literate=
    {BANG}{!}1
%    {UNDERSCORE}{\_}1
    {\\st}{Smalltalk}9 % convenience -- in case \st occurs in code
    % {'}{{\textquotesingle}}1 % replaced by upquote=true in \lstset
%    {_}{{$\leftarrow$}}1
    {>>>}{{\sep}}1
    {^}{{$\uparrow$}}1
    {~}{{$\sim$}}1
    {-}{{\sf -\hspace{-0.13em}-}}1  % the goal is to make - the same width as +
    {+}{\raisebox{0.08ex}{+}}1		% and to raise + off the baseline to match -
    {-->}{{\quad$\longrightarrow$\quad}}3
	, % Don't forget the comma at the end!
  tabsize=4
}[keywords,comments,strings]

\lstset{language=Smalltalk,
	basicstyle=\sffamily,
	keywordstyle=\color{black}\bfseries,
	% stringstyle=\ttfamily, % Ugly! do we really want this? -- on
	mathescape=true,
	showstringspaces=false,
	keepspaces=true,
	breaklines=true,
	breakautoindent=true,
	backgroundcolor=\color{source},
	lineskip={-1pt}, % Ugly hack
	upquote=true, % straight quote; requires textcomp package
	columns=fullflexible} % no fixed width fonts
% \newcommand{\ct}{\lstinline[mathescape=false,basicstyle={\sffamily\upshape}]}
\newcommand{\ct}{\lstinline[mathescape=false,backgroundcolor=\color{white},basicstyle={\sffamily\upshape}]}
\newcommand{\lct}[1]{{\textsf{\textup{#1}}}}
%\newcommand{\scat}[1]{\emph{\textsf{#1}}\xspace}
%\newcommand{\prot}[1]{\emph{\textsf{#1}}\xspace}
% NB: No argument!
\lstnewenvironment{code}[0]{%
	\lstset{%
		% frame=lines,
		frame=single,
		framerule=0pt,
		mathescape=false
	}
}{}
%\def\ignoredollar#1{}
%=============================================================
%:Reserving space
%\newcommand{\needlines}[1]{\Needspace{#1\baselineskip}}
%=============================================================
%:Indexing macros
% Macros ending with "ind" generate text as well as an index entry
% Macros ending with "index" *only* generate an index entry
\newcommand{\ind}[1]{\index{#1}#1\xspace} % plain text
\newcommand{\subind}[2]{\index{#1!#2}#2\xspace} % show #2, subindex under #1
\newcommand{\emphind}[1]{\index{#1}\emph{#1}\xspace} % emph #1
\newcommand{\emphsubind}[2]{\index{#1!#2}\emph{#2}\xspace} % show emph #2, subindex under #1
\newcommand{\patind}[1]{\index{#1@#1 (pattern)}\ct{#1}\xspace} % pattern
\newcommand{\seeindex}[2]{\index{#1|see{#2}}} % #1, see #2
%\newcommand{\boldidx}[1]{{\bf #1}} % breaks hyperlink
%\newcommand{\indmain}[1]{\index{#1}#1\xspace} 
%\newcommand{\emphsubindmain}[2]{\index{#1!#2}\emph{#2}\xspace} % subindex, main entry
%\newcommand{\subindmain}[2]{\index{#1!#2}#2\xspace} % subindex, main entry
%\newcommand{\clsindmain}[1]{\index{#1!\#@(class)}\ct{#1}\xspace} % class main
%\newcommand{\indexmain}[1]{\index{#1}} 
%=============================================================
\parskip 1ex
%=============================================================

	\pagestyle{headings}
	\setboolean{lulu}{true}
% --------------------------------------------
% A4:
%	\documentclass[a4paper,11pt,twoside]{book}
%	% $Author: oscar $
% $Date: 2009-09-13 20:58:29 +0200 (Sun, 13 Sep 2009) $
% $Revision: 29070 $
%=============================================================
% NB: documentclass must be set in main document.
% Allows book to be generated in multiple formats.
%=============================================================
%:Packages
\usepackage[T1]{fontenc}  %%%%%really important to get the code directly in the text!
\usepackage{palatino}
\usepackage{ifthen}
\usepackage{graphicx}
\graphicspath{{figures/}}
\usepackage{xspace}
\usepackage{makeidx}
\usepackage{isodateo} % enable \isodate
\usepackage{amssymb,textcomp}
%=============================================================
%:More packages
%\usepackage[english]{babel}
%\usepackage{lmodern}
%\usepackage[scaled=0.85]{helvet}
%\usepackage{microtype}
%\usepackage{theorem}
%\usepackage{float}
%\usepackage{longtable}
%\usepackage[nottoc]{tocbibind}
%\usepackage{multicol}
%\usepackage{booktabs}	% book-style tables
%\usepackage{topcapt}	% enables \topcaption
%\usepackage{multirow}
%\usepackage{tabularx}
%\usepackage{alltt}
\usepackage[usenames,dvipsnames]{color}
%\usepackage[hang]{subfigure}\makeatletter\def\p@subfigure{\thefigure\,}\makeatother
%\usepackage{rotating}
%\usepackage{enumitem}	% apb: allows more control over tags in enumerations
%\usepackage{verbatim}     % for comment environment
%\usepackage{varioref}	% for page references that work
%\usepackage{needspace}
%\usepackage[newparttoc]{titlesec}
%\usepackage{titletoc}
%\usepackage{wrapfig}
\usepackage[
	colorlinks=true,
	linkcolor=black,
	urlcolor=black,
	citecolor=black
]{hyperref}   % should come last
%=============================================================
%:URL style
\makeatletter
\def\url@leostyle{%
  \@ifundefined{selectfont}{\def\UrlFont{\sf}}{\def\UrlFont{\sffamily}}}
\makeatother
\urlstyle{leo}
%=============================================================
%:Booleans
\newboolean{lulu}
\setboolean{lulu}{false}
\newcommand{\ifluluelse}[2]{\ifthenelse{\boolean{lulu}}{#1}{#2}}
%=============================================================
%:Editorial comment macros
\newcommand{\nnbb}[2]{
  \fbox{\bfseries\sffamily\scriptsize#1}
  {\sf\small$\blacktriangleright$\textit{#2}$\blacktriangleleft$}
}
\newcommand{\on}[1]{\nnbb{Oscar}{#1}}
\newcommand{\here}{\nnbb{CONTINUE}{HERE}}
%=============================================================
%:Abbreviation macros
\newcommand{\ie}{\emph{i.e.},\xspace}
\newcommand{\eg}{\emph{e.g.},\xspace}
\newcommand{\etc}{\emph{etc.}\xspace}
\newcommand{\etal}{\emph{et al.}\xspace}
\newcommand{\straightquote}{"}
\newcommand{\sba}{\url{SquareBracketAssociates.org}\xspace}
%=============================================================
%:Patterns
% \newcommand{\pattern}[2]{\newpage\section{{\sf #1}}\label{pat:#2}}
% \newcommand{\pattern}[2]{\newpage\index{#1 (Pattern)}\section{#1}\label{pat:#2}}
\newcommand{\pattern}[2]{\cleardoublepage\index{#1 (패턴)}\section{#1}\label{pat:#2}}
\newcommand{\thumbnail}[2]{\index{#1 (패턴)}\subsection{#1}\label{pat:#2}}
\newcommand{\thumblang}[2]{\index{#1 (패턴 랭귀지)}\subsection{#1}\label{pat:#2}}
\newcommand{\variant}[1]{{\emph{#1}}\xspace}
% \newcommand{\problem}[1]{\subsection*{Problem}\emph{#1}}
\newcommand{\intent}[1]{\paragraph{의도}\emph{#1}}
\newcommand{\problem}[1]{\paragraph{문제}\emph{#1}}
\newcommand{\solution}[1]{\paragraph{해결}\emph{#1}}
\newcommand{\discussion}[0]{\paragraph{토론}}
\newcommand{\cmd}[1]{{\tt #1}\xspace}
%=============================================================
%:Environments
\newenvironment{bulletlist}{\begin{itemize}\setlength{\itemsep}{0ex}}
{\end{itemize}}
%=============================================================
%:Cross reference macros
\newcommand{\chalabel}[1]{\label{cha:#1}}
\newcommand{\seclabel}[1]{\label{sec:#1}}
\newcommand{\figlabel}[1]{\label{fig:#1}}
\newcommand{\tablabel}[1]{\label{tab:#1}}
\newcommand{\rulelabel}[1]{\label{rule:#1}}
\newcommand{\eglabel}[1]{\label{eg:#1}}
\newcommand{\scrlabel}[1]{\label{scr:#1}}
\newcommand{\mthlabel}[1]{\label{mth:#1}}
\newcommand{\clslabel}[1]{\label{cls:#1}}
\newcommand{\faqlabel}[1]{\label{faq:#1}}
%\newcommand{\charef}[1]{Chapter~\ref{cha:#1}\xspace}
%\newcommand{\secref}[1]{Section~\ref{sec:#1}\xspace}
\newcommand{\figref}[1]{Figure~\ref{fig:#1}\xspace}
% \newcommand{\patpgref}[2]{\hyperref[pat:#2]{\sf #1} [p.~\pageref{pat:#2}]\xspace}
\newcommand{\patpgref}[2]{\index{#1 (Pattern)}\hyperref[pat:#2]{#1} [p.~\pageref{pat:#2}]\xspace}
\newcommand{\patlangpgref}[2]{\index{#1 (Pattern language)}\hyperref[pat:#2]{#1} [p.~\pageref{pat:#2}]\xspace}
% \newcommand{\patref}[2]{\hyperref[pat:#2]{\sf #1}\xspace}
\newcommand{\patref}[2]{\index{#1 (Pattern)}\hyperref[pat:#2]{#1}\xspace}
\newcommand{\patlangref}[2]{\index{#1 (Pattern language)}\hyperref[pat:#2]{#1}\xspace}
% \newcommand{\charef}[2]{\hyperref[cha:#2]{\underline{\sf #1}}\xspace}
% \newcommand{\charef}[2]{\hyperref[cha:#2]{\sf #1}\xspace}
\newcommand{\charef}[2]{\index{#1 (Pattern cluster)}\hyperref[cha:#2]{#1}\xspace}
% \newcommand{\chapgref}[2]{\hyperref[cha:#2]{\sf #1} [p.~\pageref{cha:#2}]\xspace}
\newcommand{\chapgref}[2]{\index{#1 (Pattern cluster)}\hyperref[cha:#2]{#1} [p.~\pageref{cha:#2}]\xspace}
%\newcommand{\Figref}[1]{Figure~\ref{fig:#1}\xspace}
%\newcommand{\appref}[1]{Appendix~\ref{app:#1}\xspace}
%\newcommand{\tabref}[1]{Table~\ref{tab:#1}\xspace}
%\newcommand{\ruleref}[1]{\ref{rule:#1}\xspace}
%\newcommand{\egref}[1]{example~\ref{eg:#1}\xspace}
%\newcommand{\Egref}[1]{Example~\ref{eg:#1}\xspace}
%\newcommand{\scrref}[1]{script~\ref{scr:#1}\xspace}
%\newcommand{\Scrref}[1]{Script~\ref{scr:#1}\xspace}
%\newcommand{\tscrref}[1]{the script~\ref{scr:#1}\xspace}
%\newcommand{\Tscrref}[1]{The script~\ref{scr:#1}\xspace}
%\newcommand{\mthref}[1]{method~\ref{mth:#1}\xspace}
%\newcommand{\mthsref}[1]{methods~\ref{mth:#1}\xspace}
%\newcommand{\Mthref}[1]{Method~\ref{mth:#1}\xspace}
%\newcommand{\tmthref}[1]{the method~\ref{mth:#1}\xspace}
%\newcommand{\Tmthref}[1]{The method~\ref{mth:#1}\xspace}
%\newcommand{\clsref}[1]{class~\ref{cls:#1}\xspace}
%\newcommand{\tclsref}[1]{the class~\ref{cls:#1}\xspace}
%\newcommand{\Tclsref}[1]{The class~\ref{cls:#1}\xspace}
%=============================================================
%:Page Layout
\setlength{\headsep}{1cm}
%=============================================================
%:Menu item macro
%\definecolor{lightgray}{gray}{0.89}
%\newcommand{\menu}[1]{{%
%	\setlength{\fboxsep}{0pt}%
%	\colorbox{lightgray}{{{\upshape\sffamily\strut \,#1\,}}}}}
%\newcommand{\go}{\,$\triangleright$\,}
%\newcommand{\short}[1]{\mbox{{\sc cmd}\hspace{0.08em}--\hspace{0.09em}#1}\xspace}
%\newcommand{\button}[1]{{%
%	\setlength{\fboxsep}{0pt}%
%	\fbox{{\upshape\sffamily\strut \,#1\,}}}}
%\newcommand{\toolsflap}{\textit{Tools} flap\xspace}
%=============================================================
%:Section depth
%\setcounter{secnumdepth}{2}
%
%\DeclareGraphicsExtensions{.pdf, .jpg, .png}
%=============================================================
%:PDF setup
\hypersetup{
   pdftitle={Object-Oriented Reengineering Patterns},
   pdfauthor={Serge Demeyer, St\'ephane Ducasse, Oscar Nierstrasz},
   pdfkeywords={Reengineering, Object-Oriented Programming, Patterns},
   pdfsubject={Computer Science}
}
%=============================================================
%:Page layout and appearance
%\renewcommand{\chaptermark}[1]{\markboth{#1}{}}
%\renewcommand{\sectionmark}[1]{\markright{\thesection\ #1}}
%\renewpagestyle{plain}[\small\itshape]{%
%	\setheadrule{0pt}%
%	\sethead[][][]{}{}{}%
%	\setfoot[][][]{}{}{}}
%\renewpagestyle{headings}[\small\itshape]{%
%	\setheadrule{0pt}%
%	\setmarks{chapter}{section}%
%	\sethead[\thepage][][\chaptertitle]{\sectiontitle}{}{\thepage}%
%	\setfoot[][][]{}{}{}}
%=============================================================
%:Title section setup and TOC numbering depth
%\setcounter{secnumdepth}{1}
%\setcounter{tocdepth}{1}
%\titleformat{\part}[display]{\centering}{\huge\partname\ \thepart}{1em}{\Huge\textbf}[]
%\titleformat{\chapter}[display]{}{\huge\chaptertitlename\ \thechapter}{1em}{\Huge\raggedright\textbf}[]
%\titlecontents{part}[3pc]{%
%		\pagebreak[2]\addvspace{1em plus.4em minus.2em}%
%		\leavevmode\large\bfseries}
%	{\contentslabel{3pc}}{\hspace*{-3pc}}
%	{}[\nopagebreak]
%\titlecontents{chapter}[3pc]{%
%		\pagebreak[0]\addvspace{1em plus.2em minus.2em}%
%		\leavevmode\bfseries}
%	{\contentslabel{3pc}}{}
%	{\hfill\contentspage}[\nopagebreak]
%\dottedcontents{section}[3pc]{}{3pc}{1pc}
%\dottedcontents{subsection}[3pc]{}{0pc}{1pc}
%\let\origdoublepage\cleardoublepage
%\newcommand{\clearemptydoublepage}{%
%  \clearpage
%  {\pagestyle{empty}\origdoublepage}}
%\let\cleardoublepage\clearemptydoublepage % see http://www.tex.ac.uk/cgi-bin/texfaq2html?label=patch
%=============================================================
%:Listings package configuration
\newcommand{\caret}{\makebox{\raisebox{0.4ex}{\footnotesize{$\wedge$}}}}
% \newcommand{\escape}{{\sf \textbackslash}}
\definecolor{source}{gray}{0.95}
\usepackage{listings}
\lstdefinelanguage{Smalltalk}{
  morestring=[d]',
% Adapt this to other languages!
%  morecomment=[s]{"}{"},
  alsoletter={\#:},
  %escapechar={!},
  literate=
    {BANG}{!}1
%    {UNDERSCORE}{\_}1
    {\\st}{Smalltalk}9 % convenience -- in case \st occurs in code
    % {'}{{\textquotesingle}}1 % replaced by upquote=true in \lstset
%    {_}{{$\leftarrow$}}1
    {>>>}{{\sep}}1
    {^}{{$\uparrow$}}1
    {~}{{$\sim$}}1
    {-}{{\sf -\hspace{-0.13em}-}}1  % the goal is to make - the same width as +
    {+}{\raisebox{0.08ex}{+}}1		% and to raise + off the baseline to match -
    {-->}{{\quad$\longrightarrow$\quad}}3
	, % Don't forget the comma at the end!
  tabsize=4
}[keywords,comments,strings]

\lstset{language=Smalltalk,
	basicstyle=\sffamily,
	keywordstyle=\color{black}\bfseries,
	% stringstyle=\ttfamily, % Ugly! do we really want this? -- on
	mathescape=true,
	showstringspaces=false,
	keepspaces=true,
	breaklines=true,
	breakautoindent=true,
	backgroundcolor=\color{source},
	lineskip={-1pt}, % Ugly hack
	upquote=true, % straight quote; requires textcomp package
	columns=fullflexible} % no fixed width fonts
% \newcommand{\ct}{\lstinline[mathescape=false,basicstyle={\sffamily\upshape}]}
\newcommand{\ct}{\lstinline[mathescape=false,backgroundcolor=\color{white},basicstyle={\sffamily\upshape}]}
\newcommand{\lct}[1]{{\textsf{\textup{#1}}}}
%\newcommand{\scat}[1]{\emph{\textsf{#1}}\xspace}
%\newcommand{\prot}[1]{\emph{\textsf{#1}}\xspace}
% NB: No argument!
\lstnewenvironment{code}[0]{%
	\lstset{%
		% frame=lines,
		frame=single,
		framerule=0pt,
		mathescape=false
	}
}{}
%\def\ignoredollar#1{}
%=============================================================
%:Reserving space
%\newcommand{\needlines}[1]{\Needspace{#1\baselineskip}}
%=============================================================
%:Indexing macros
% Macros ending with "ind" generate text as well as an index entry
% Macros ending with "index" *only* generate an index entry
\newcommand{\ind}[1]{\index{#1}#1\xspace} % plain text
\newcommand{\subind}[2]{\index{#1!#2}#2\xspace} % show #2, subindex under #1
\newcommand{\emphind}[1]{\index{#1}\emph{#1}\xspace} % emph #1
\newcommand{\emphsubind}[2]{\index{#1!#2}\emph{#2}\xspace} % show emph #2, subindex under #1
\newcommand{\patind}[1]{\index{#1@#1 (pattern)}\ct{#1}\xspace} % pattern
\newcommand{\seeindex}[2]{\index{#1|see{#2}}} % #1, see #2
%\newcommand{\boldidx}[1]{{\bf #1}} % breaks hyperlink
%\newcommand{\indmain}[1]{\index{#1}#1\xspace} 
%\newcommand{\emphsubindmain}[2]{\index{#1!#2}\emph{#2}\xspace} % subindex, main entry
%\newcommand{\subindmain}[2]{\index{#1!#2}#2\xspace} % subindex, main entry
%\newcommand{\clsindmain}[1]{\index{#1!\#@(class)}\ct{#1}\xspace} % class main
%\newcommand{\indexmain}[1]{\index{#1}} 
%=============================================================
\parskip 1ex
%=============================================================

%	\usepackage{a4wide}
% --------------------------------------------
	\begin{document}
	\renewcommand{\nnbb}[2]{} % Disable editorial comments
	\sloppy
\fi
%=================================================================
\chapter{마이그레이션 전략}
\chalabel{MigrationStrategies}

\on{Put these in the right places:}

\label{pat:AlwaysHaveARunningVersion}

리엔지니어링 프로젝트가 잘 진행되고 있다. 레거시 시스템을 잘 이해하고 있으며 \patpgref{진화 활성화를 위한 테스트 작성하기}{WriteTestsToEnableEvolution}을 시작했다. \charef{방향 설정}{SettingDirection} 프로세스를 거쳤고, \patpgref{가장 가치 있는 것 먼저 하기}{MostValuableFirst}를 처리하기로 결정했다.

새 시스템이 사용자들에게 받아들여질지 어떻게 확신할 수 있을까? 이전 시스템을 사용하는 동안 새 시스템으로 어떻게 마이그레이션할 수 있을까? 새 시스템이 완성되기 전에 어떻게 테스트하고 평가할 수 있을까?

\subsection*{포스: 주요한 요구사항}

\begin{bulletlist}
\item 빅뱅 마이그레이션(Big-bang migration)은 실패 위험이 높다.

\item 한 번에 너무 많은 변경 사항을 도입하면 사용자가 거리감을 느낄 수 있다.

\item 지속적인 피드백은 어렵고 비용이 많이 들 수 있지만 궤도를 유지하는 데 도움이 된다.

\item 사용자는 업무를 완수해야 하며 불완전한 솔루션때문에 인해 방해받고 싶지는 않다.

\item 레거시 데이터는 시스템을 사용하는 동안 보존되어야 한다.
\end{bulletlist}

\subsection*{개요}

레거시 시스템을 리엔지니어링한 다음 배포하는 것만으로는 충분하지 않다. 사실 이렇게 시도하면 (새로운 영역에서 대규모 폭포수 프로젝트가 종종 실패하는 것과 같은 이유로) 반드시 실패할 것이다. 새로운 솔루션을 점진적으로 도입하여 사용자들의 신뢰와 협조를 얻을 준비가 되어 있어야 하며, 기존 시스템에서 새 시스템으로 \emph{아직 배포 중인 동안} 점진적이고 고통 없이 마이그레이션하는 전략을 채택해야 한다.

\begin{figure}
\begin{center}
\includegraphics[width=\textwidth]{MigrationMap}
\caption{레거시 시스템을 마이그레이션하는 방법, 이유 및 대상}
\figlabel{MigrationMap}
\end{center}
\end{figure}

이 클러스터의 핵심 메시지는 \patref{시스템 점진적 마이그레이션하기}{MigrateSystemsIncrementally}이다. 그러나 이것은 말처럼 쉽지 않다. 그림 25에서 \patref{시스템 점진적 마이그레이션하기}{MigrateSystemsIncrementally}를 수행하려면 수많은 다른 패턴을 고려해야 한다는 것을 알 수 있다. 시스템 마이그레이션에 대한 방대한 문헌이 존재하기 때문에 이 주제를 자세히 다루지는 않겠다. 그러나 객체 지향 레거시 시스템을 리엔지니어링하는 데 가장 중요하다고 생각되는 패턴을 선택하고 주요 요점을 요약했다. 적절한 경우에 대해 독자는 추가 정보 출처를 찾을 수 있다.

이 클러스터의 중심 패턴은 \patref{시스템 점진적 마이그레이션}{MigrateSystemsIncrementally}이지만, 핵심 동기는 \patref{사용자 참여시키기}{InvolveTheUsers}와 \patref{자신감 구축하기}{BuildConfidence}에서 제공된다. 이 처음 세 가지 패턴은 위험을 최소화하고 성공 확률을 높이기 위한 기본 패턴이다.

\begin{bulletlist}
\item \patref{사용자 참여시키기}{InvolveTheUsers}는 전체 리엔지니어링 프로세스에 사용자를 긴밀하게 참여시키고, 중간 결과물을 사용하게 하고, 강력한 지원을 제공함으로써 사용자가 새로운 시스템을 받아들일 가능성을 높인다. 단계별로 \patref{시스템 점진적 마이그레이션하기}{MigrateSystemsIncrementally}와 \patref{자신감 구축하기}{BuildConfidence}를 사용하면 더 쉽게 달성할 수 있다.

\item \patref{자신감 구축하기}{BuildConfidence}는 사용자에게 가치 있는 결과를 정기적으로 제공함으로써 회의론과 의심을 극복하는 데 도움이 된다. 

\item \patref{시스템 점진적 마이그레이션하기}{MigrateSystemsIncrementally}는 기존 시스템을 점진적이고 점진적으로 새 시스템으로 대체할 것을 권장한다. 그런 다음 진행하면서 새로운 결과를 통합할 수 있으므로 \patref{자신감 구축하기}{BuildConfidence} 및 \patref{사용자 참여시키기}{InvolveTheUsers}에 도움이 된다.
\end{bulletlist}

다음 사례도 준수하지 않으면 \patref{시스템 점진적 마이그레이션하기}{MigrateSystemsIncrementally}를 수행하기가 매우 어렵다.

\begin{bulletlist}
\item \patref{목표 솔루션 프로토타입하기}{PrototypeTheTargetSolution}을 사용하여 새 아키텍처와 새로운 기술적 위험을 테스트한다. 이미 실행 중인 시스템이 있기 때문에 프로토타입이 필요 없다고 생각하기 쉽지만, 이는 거의 항상 실수이다.

\item \patref{실행 버전 항상 보유하기}{AlwaysHaveARunningVersion}는 변경 사항을 자주 통합하여 동기화 상태를 유지하는 데 도움이 된다.

\item \patref{변경할 때마다 회귀 테스트하기}{RegressionTestAfterEveryChange}는 실행 중이던 모든 것이 계속 실행되도록 함으로써 \patref{실행 버전 항상 보유하기}{AlwaysHaveARunningVersion}에 도움이 된다. 여기에는 \patpgref{진화 활성화를 위한 테스트 작성하기}{WriteTestsToEnableEvolution}가 전제되어 있다.
\end{bulletlist}

상황에 따라 \patref{시스템 점진적 마이그레이션하기}{MigrateSystemsIncrementally}에 도움이 될 수 있는 다양한 방법이 있다.

\begin{bulletlist}
\item \patref{뉴 타운으로 가는 다리 만들기}{MakeABridgeToTheNewTown}는 (데이터) ``다리''의 은유를 도입하여 레거시 컴포넌트에서 대체 컴포넌트로 데이터를 점진적으로 마이그레이션하는 동시에 두 요소가 함께 실행될 수 있도록 한다. 모든 데이터가 전송되면 레거시 컴포넌트를 폐기할 수 있다.

\item \patref{올바른 인터페이스 제시하기}{PresentTheRightInterface}는 이전 기능을 래핑하여 실제로 원하는 추상화를 내보내 목표 시스템을 점진적으로 개발하는 데 도움이 된다.

\item \patref{퍼블릭 인터페이스와 게시된 인터페이스 구분하기}{DistinguishPublicFromPublishedInterface}는 리엔지니어링 팀 내에서 병렬 개발을 용이하게 하기 위해 안정적인 (퍼블릭) 인터페이스(public interface)와 불안정한 (게시된) 인터페이스(published interface)를 구분한다.

\item \patref{폐기된 인터페이스 지원 중단하기}{DeprecateObsoleteInterfaces}를 사용하면 클라이언트를 즉시 무효화하지 않고도 사용되지 않는 인터페이스를 정상적으로 폐기할 수 있다.
\end{bulletlist}

마지막으로, 다음 두 가지 방법은 급진적이지만 불필요한 변경을 피하는 데 도움이 될 수 있다.


\begin{bulletlist}
\item  \patref{친숙도 보존하기}{ConserveFamiliarity}는 사용자가 어색함을 느낄 수 있는 급격한 인터페이스 변경을 도입하지 않도록 경고한다.

\item \patpgref{최적화하기 전에 프로파일러 사용하기}{UseProfilerBeforeOptimizing}은 문제가 있음을 입증하고 문제의 원인을 정확히 파악하여 성능 문제를 고려하도록 한다.
\end{bulletlist}

%=================================================================
%:PATTERN -- {Involve the Users}
\pattern{사용자 참여시키기}{InvolveTheUsers}

\ind{고객과 관계맺기}\emph{(Engage Customers)로도 알려져 있다.} \cite{Copl95d}

\intent{모든 단계에서 사용자를 참여시켜 변경 사항의 수용을 극대화한다.}

\subsection*{문제}

사용자가 리엔지니어링된 시스템을 받아들일 것이라고 어떻게 확신할 수 있는가?

\emph{이 문제는 다음과 같은 이유로 어렵다.}

\begin{bulletlist}
\item 이전 시스템이 작동한다. 투박하지만 사용자들은 작동 방식을 알고 있고 문제를 해결하는 방법을 알고 있다.

\item 사람들은 자신의 삶을 더 편리하게 만들어 주지 않는 한 새로운 것을 배우는 것을 싫어한다.

\item 시스템 개선에 필요한 사항에 대한 사용자의 인식은 시스템이 발전함에 따라 변화하는 경향이 있다.

\item 사용자는 종이 위에만 있는 설계를 평가하는 것은 어려워 한다.

\item 사용할 준비가 되지 않은 새로운 시스템을 사용자가 좋아하게 만드는 것은 어렵다.
\end{bulletlist}

\emph{그러나 이 문제를 해결할 수 있는 이유는 다음과 같다.}

\begin{bulletlist}
\item 사용자는 자신의 요구가 진지하게 해결되고 있다고 판단되면 새로운 솔루션을 시도할 것이다.

\item 사용자는 유용한 정보를 제공하면 피드백을 줄 것이다.
\end{bulletlist}

\subsection*{해결}

사용자를 새로운 개발에 직접 참여시키고 새로운 시스템을 사용할 수 있도록 긴밀하게 지원하자.

\subsubsection*{단계}

사용자가 우선순위가 어디에 있는지 말할 수 있도록 하자. \patpgref{가장 가치 있는 것 먼저 하기}{MostValuableFirst}로 시작하자. 우선순위를 정기적으로 전달할 수 있는 작은 단계로 나누면 \patpgref{자신감 구축하기}{BuildConfidence}와 같이 사용할 수 있다.

사용자와 개발자 간의 접촉을 장려할 수 있는 환경을 조성하자. 물리적 위치가 중요하다.

정기적으로 중간 결과물을 전달하고 피드백을 받을 수 있는 간단한 절차를 마련하자. 초기 프로토타입은 특히 위험한 신기술이나 접근 방식을 평가하는 데 도움이 될 수 있다. 좋은 전략은 사용자가 새 시스템이 구축되는 대로 사용할 수 있도록 \patpgref{시스템 점진적 마이그레이션하기}{MigrateSystemsIncrementally}를 수행하는 것이다. 사용자가 느끼는 친숙도를 떨어 뜨리지 하기 위해 \patpgref{친숙도 보존하기}{ConserveFamiliarity}를 설정해야 한다.

\subsection*{트레이드오프}

\subsubsection*{장점}

\begin{bulletlist}
\item 요구사항이 지속적으로 검증되고 업데이트되므로 올바른 방향으로 나아갈 가능성이 높아진다.

\item 사용자가 유용한 결과를 얻고 있고 지원을 받고 있다고 느끼면 유용한 피드백을 제공하기 위해 더 많은 노력을 기울일 것이다.

\item 사용자가 프로젝트 전반에 걸쳐 참여하므로 프로젝트 후반에 특별한 교육 세션이 필요하지 않다.
\end{bulletlist}

\subsubsection*{단점}

\begin{bulletlist}
\item 개발자는 사용자를 지원하는 것이 시스템 리엔지니어링 작업에 방해가 된다고 느낄 수 있다.

\index{유든, 에드워드}
\item 사용자 참여가 성공하면 기대치가 높아지고 팀에 추가적인 압박이 가해진다. 예를 들어, 유든은 프로토타입은 기대치를 지나치게 높일 수 있으며 아직 작동하지 않는 부분을 항상 명확히 해야 한다고 이야기 한다 \cite{Your97a}.
\end{bulletlist}

\subsubsection*{어려움}

\begin{bulletlist}
\item 결과를 보여 줄 수 있기 전까지는 사용자를 참여를 시작하기 어려울 수 있다.

\item 모든 사용자를 참여시킬 수는 없으며, 포함되지 않은 사용자는 소외감을 느낄 수 있다.
\end{bulletlist}

\subsection*{근거}

실제 고객의 요구 사항을 다루려면 피드백 루프가 필요하다. 사용자를 참여시키고 지원함으로써 이러한 피드백 루프가 활성화 되도록 할 수 있다.

\index{코플리언, 제임스}
\emph{```제품 품질 유지(maintaining product quality)'가 여기서 해결해야 할 문제가 아니라는 점에 유의하자. 제품 품질은 고객 만족의 한 요소일 뿐이다.''}라고 코플리언은 지적했다. \cite{Copl95d}

\subsection*{관련 패턴}

이 클러스터의 거의 모든 패턴은 \patref{사용자 참여시키기}{InvolveTheUsers}를 지원한다. \patref{시스템 점진적 마이그레이션 하기}{MigrateSystemsIncrementally}를 통해 사용자가 리엔지니어링 중인 시스템을 함께 작업하도록 하여 \patref{자신감 구축하기}{BuildConfidence}를 구현한다.

\ind{계획 게임}\emph{(Planning Game)} \cite{Beck01a}는 반복적으로 스토리를 파악하고, 비용을 추산하고, 출시할 스토리를 제품에 적용함으로써 \patref{사용자 참여시키기}{InvolveTheUsers}를 구현하는 효과적인 기법이다.

%=================================================================
%:PATTERN -- {Build Confidence}
\pattern{자신감 구축하기}{BuildConfidence}

\intent{규칙적으로 성과를 보여줌으로써 전반적인 성공 가능성을 높인다.}

\subsection*{문제}

모든 종류의 소프트웨어 프로젝트에 대해 고객과 팀원이 종종 갖는 높은 수준의 회의감(skepticism)을 어떻게 극복할 수 있는가?

\emph{이 문제는 다음과 같은 이유로 어렵다.}

\begin{bulletlist}
\item 요구 사항을 충족하고 제시간에 완료되며 예산 범위 내에서 진행되는 소프트웨어 프로젝트는 거의 없다. 대부분의 프로젝트에 수반되는 회의주의는 쉽게 패배주의(defeatism)로 이어질 수 있으며, 프로젝트는 자기 충족적 예언처럼 실패할 수 있다.

\item 사용자가 진정으로 원하거나 필요로 하는 것을 거의 얻지 못한다.

\item 레거시 시스템을 실제로 잘 유지할 수 있다고 사용자나 팀원들을 설득하기 어려울 수 있다.
\end{bulletlist}

\emph{그러나 이 문제를 해결할 수 있는 이유는 다음과 같다.}

\begin{bulletlist}
\item 모든 문제를 한 번에 해결할 필요는 없습니다.
\end{bulletlist}

\subsection*{해결}

가능한 한 빨리 긍정적인 결과를 보여줌으로써 긍정적인 분위기를 조성하고, 정기적으로 계속 그렇게 하자.

\subsubsection*{단계}

짧은 간격을 두고 새로운 결과를 전달하자. 각 단계에서 실질적인 가치를 보여줄 수 있는 최소한의 결과가 무엇인지 사용자와 함께 합의하자.

\subsection*{트레이드오프}

\subsubsection*{장점}

\begin{bulletlist}
\item 사용자와 개발자 모두 실제 진행 상황을 측정할 수 있다.

\item 작은 단계의 비용을 더 쉽게 추정할 수 있다.
\end{bulletlist}

\subsubsection*{단점}

\begin{bulletlist}
\item 사용자와 자주 동기화하는 데 시간이 걸린다.

\item 사용자는 새 시스템을 이전 시스템과 함께 사용하는 데 필요한 추가 작업을 거부할 수 있다.

\item 프로젝트 초기에 좋은 결과를 보여 주는 데 성공하면 기대치가 너무 높아질 수 있다.
\end{bulletlist}

\subsubsection*{어려움}

\begin{bulletlist}
\item 일부 요구 사항은 특히 시스템의 아키텍처 변경을 수반하는 경우 작은 단계로 나누기 어려울 수 있다.

\item 리엔지니어링 팀은 가장 중요한 정보 소스 중 하나인 원래 시스템의 개발자를 소외시키지 않도록 주의해야 한다.

\item 사용자를 설득하는 것만으로는 충분하지 않으며 경영진의 동의를 얻는 데도 신경을 써야 한다. 작은 단계로 경영진을 설득하기는 어렵다. 정기적으로 대규모 데모를 계획하자.
\end{bulletlist}

\subsection*{근거}

작은 단계를 밟아나가면 개별 단계가 실패할 위험을 줄일 수 있다. 긍정적인 결과를 자주 얻으면 자신감을 키우는 데 도움이 된다. 같은 맥락에서 \ind{익스트림 프로그래밍}(Extreme Programming)은 \ind{소규모 릴리즈}(Small Releases)를 지지한다. \cite{Beck00a}. 부정적인 결과라도 진행 상황을 모니터링하고 상황을 더 잘 이해하는 데 도움이 되므로 자신감을 키우는 데 도움이 된다.

\subsection*{관련 패턴}

\patref{목표 솔루션 프로토타입 만들기}{PrototypeTheTargetSolution}와 \patref{뉴 타운으로 가는 다리 만들기}{MakeABridgeToTheNewTown}를 사용하면 작은 단계로 결과를 쉽게 보여줄 수 있다. 

\patref{사용자 참여시키기}{InvolveTheUsers}를 사용하면 \patref{자신감 구축하기}{BuildConfidence}를 더 쉽게 수행할 수 있다. 

%=================================================================
%:PATTERN -- {Migrate Systems Incrementally}
\pattern{시스템 점진적 마이그레이션하기}{MigrateSystemsIncrementally}

\ind{치킨 리틀}(Chicken Little)로도 \emph{알려져 있다}. \cite{Brod95a}

\intent{기능을 빈번한 증분 개념으로 배포함으로써 빅뱅 리엔지니어링의 복잡성과 리스크를 피하자.}

\subsection*{문제}

새 시스템 배포는 언제 계획해야 하는가?

\emph{이 문제는 다음과 같은 이유로 어렵다.}

\begin{bulletlist}
\item 프로젝트는 ``빅뱅'' 요구 사항 사양을 미리 작성하여 대규모로 계획하고 자금을 지원하는 경우가 많다.

\item 실제 요구사항은 뒤늦게야 명확해지는 경우가 많다. 특히 처음부터 완벽하게 작동하지 않는다면 사용자는 익숙한 것과 근본적으로 다른 새로운 시스템 사용을 거부할 것이다.

\item 새 시스템을 배포하는 데 시간이 오래 걸릴수록 사용자 피드백을 받기 위해 더 오래 기다려야 한다.

\item 불완전한 시스템을 배포할 수 없다. 사용자는 불완전한 솔루션에 시간을 낭비하고 싶어하지 않는다.
\end{bulletlist}

\emph{그러나 이 문제를 해결할 수 있는 이유는 다음과 같다.}

\begin{bulletlist}
\item 확장 및 수정할 수 있는 동작하는 시스템이 있다.
\end{bulletlist}

\subsection*{해결}

가능한 한 빨리 레거시 시스템의 첫 번째 \emph{업데이트(update)}를 배포하고 대상 시스템으로 점진적으로 마이그레이션하자.

\subsubsection*{단계}

\begin{bulletlist}
\item 레거시 시스템을 여러 부분으로 분해한다.

\item 한 번에 하나씩 처리할 부분을 선택한다.

\item 해당 부분과 해당 부분에 의존하는 부분에 대한 테스트를 수행한다.

\item 레거시 컴포넌트를 래핑, 리엔지니어링 또는 교체하기 위한 적절한 조치를 취한다.

\item 업데이트된 컴포넌트를 배포하고 피드백을 받는다.

\item 반복한다.
\end{bulletlist}

\subsection*{트레이드오프}

\subsubsection*{장점}

\begin{bulletlist}
\item 사용자 피드백을 조기에 받고 \patref{자신감 구축하기}{BuildConfidence}를 수행한다.

\item 문제가 발생하면 즉시 알 수 있다.

\item 사용자는 새 시스템이 구축되는 동안 학습한다.

\item 시스템이 항상 배포된다.

\item 시스템은 항상 테스트 중이므로 테스트를 건너뛸 수 없다.
\end{bulletlist}

\subsubsection*{단점}

\begin{bulletlist}
\item 시스템을 변경하는 동안 시스템을 계속 실행하려면 더 많은 노력을 기울어야 한다.
\end{bulletlist}

\subsubsection*{어려움}

\begin{bulletlist}
\item 새로운 아키텍처로 마이그레이션하는 것이 어려울 수 있다. 새 아키텍처를 적용하기 위해 \patref{목표 솔루션 프로토타입하기}{PrototypeTheTargetSolution}를 사용할 수 있다, 그리고 기본 구성 요소를 마이그레이션하는 동안 레거시 인터페이스를 숨기려면 \patref{올바른 인터페이스 제시하기}{PresentTheRightInterface}를 이전 시스템에 적용할 수 있다.

\item 실행 중인 시스템을 변경하는 것은 위험하다. 반드시 \patref{변경할 때마다 회귀 테스트하기}{RegressionTestAfterEveryChange}를 수행하자. 
\end{bulletlist}

\subsection*{근거}

실행 중인 시스템에서 최고의 사용자 피드백을 얻을 수 있다. 사용자는 매일 사용하는 시스템에서 피드백을 제공하는 더 많은 동기를 얻고 참여할 수 있다.

\subsection*{알려진 용도}

\emphind{Migrating Legacy Systems} \cite{Brod95a}에서는 이 패턴을 '\ind{치킨 리틀}(Chiceken Little)'이라는 이름으로 소개한다(점진적으로 마이그레이션한다는 것은 '치킨 리틀 단계를 밟는다'는 뜻이다). 이 책에서는 점진적 마이그레이션을 위한 전략과 기법에 대해 자세히 설명한다.

\subsection*{관련 패턴}

\patpgref{가장 가치 있는 것 먼저 하기}{MostValuableFirst}를 적용하여 먼저 작업할 레거시 컴포넌트를 선택한다. 아키텍처 무결성을 유지하기 위해 \patpgref{내비게이터 지정하기}{AppointANavigator}를 적용한다. 

마이그레이션할 때 \patpgref{진화 활성화를 위한 테스트 작성하기}{WriteTestsToEnableEvolution}, \patpgref{기본 테스트를 증가시키기}{GrowYourTestBaseIncrementally}를 실행하자. 레거시 컴포넌트를 리엔지니어링하거나 교체할 때 항상 테스트를 다시 작성할 필요가 없도록 \patpgref{구현이 아닌 인터페이스 테스트하기}{TestTheInterfaceNotTheImplementation}을 사용하자. \patref{변경할 때마다 회귀 테스트하기}{RegressionTestAfterEveryChange}를 사용하면 \patpgref{실행 버전 항상 보유하기}{AlwaysHaveARunningVersion}을 사용할 수 있다.

리엔지니어링하거나 교체할 의도가 없는 레거시 컴포넌트에 대해서는 \patref{올바른 인터페이스 제시하기}{PresentTheRightInterface}를 적용하는 것이 좋다.

교체하려는 레거시 컴포넌트에서 데이터를 마이그레이션해야 하는 경우 \patpgref{뉴 타운으로 가는 다리 만들기}{MakeABridgeToTheNewTown}을 적용할 수 있다.

%=================================================================
%:PATTERN -- {Prototype the Target Solution}
\pattern{목표 솔루션 프로토타입하기n}{PrototypeTheTargetSolution}

\intent{프로토타입을 만들어 새 대상 솔루션으로 마이그레이션할 위험을 평가한다.}

\subsection*{문제}

새 목표 시스템에 대한 아이디어가 효과가 있는지 어떻게 알 수 있는가?

\emph{이 문제는 다음과 같은 이유로 어렵다.}

\begin{bulletlist}
\item 작동 중인 시스템을 급격하게 변경하는 것은 위험하다.

\item 설계 변경이 기존 기능에 어떤 영향을 미칠지 예측하기 어려울 수 있다.

\item 작동하는 솔루션이 테스트되지 않은 솔루션보다 더 신뢰할 수 있다.
\end{bulletlist}

\emph{그러나 이 문제를 해결할 수 있는 이유는 다음과 같다.}

\begin{bulletlist}
\item 새로운 아이디어를 테스트하기 위해 레거시 시스템 전체를 리엔지니어링할 필요는 없다.
\end{bulletlist}

\subsection*{해결}

새로운 개념을 위해 개발한 프로토타입을 가지고 새롭고 창발하는 요구 사항에 대해 평가하자.

\subsubsection*{단계}

\begin{bulletlist}
\item 리엔지니어링 프로젝트의 가장 큰 기술적 위험을 파악하자. 일반적으로 다음과 같은 우려 사항이 있다.

\begin{bulletlist}
\item 새로운 시스템 아키텍처의 선택
\item 레거시 데이터의 새 시스템으로의 마이그레이션
\item 새로운 기술 또는 플랫폼을 통한 적절한 성능 또는 성능 향상(예: 특정 트랜잭션 처리량을 달성할 수 있음을 입증)
\end{bulletlist}

\index{버려질 프로토타입}
\index{탐색용 프로토타입}
\index{진화적 프로토타입}
\item 기술 옵션의 실현 가능성을 순수하게 평가하기 위한 탐색용 프로토타입(exploratory prototype) 즉, 버려질 프로토타입(throwaway prototype)을 구현할지 아니면 최종적으로 새로운 목표 시스템으로 진화할 진화적 프로토타입(evolutionary prototype)을 구현할지 결정합니다.

\begin{bulletlist}
\item 탐색용 프로토타입(exploratory prototype)은 매우 정확한 질문에 답할 수 있도록 설계되어야 한다. 이러한 질문은 새 플랫폼이 레거시 시스템에서 설정한 성능 제약을 충족할 수 있는지와 같은 순전히 기술적인 질문일 수도 있고, 사용자의 참여와 평가가 필요한 사용성 질문일 수도 있다. 탐색용 프로토타입은 (이 프로토타입이 제공하는 답변이 새 시스템에 영향을 미치기는 하지만) 다른 문제나 질문을 해결하기 위해 설계될 필요가 없으며 마이그레이션헐 시스템의 일부가 아니다.

\item 반면에 진화형 프로토타입(evolutionary prototype)은 궁극적으로 레거시 구성 요소를 대체하기 위한 것이므로 목표 아키텍처를 반영해야 한다. 새로운 아키텍처는 레거시 서비스를 가장 적절하게 지원할 뿐만 아니라 레거시 솔루션의 유용성을 제한하는 장애물도 극복한다. 프로토타입은 이러한 리스크에 먼저 대응할 수 있도록 설계되어야 한다.
\end{bulletlist}
\end{bulletlist}

\subsection*{트레이드오프}

\subsubsection*{장점}

\begin{bulletlist}
\item 레거시 시스템의 모든 기능을 구현할 필요가 없으므로 프로토타입을 빠르게 구축할 수 있다.

\item 프로토타입을 실행하기 위해 레거시 시스템의 일부를 해킹할 수 있다.

\item 목표 시스템에 대한 아이디어가 타당하다면 빠르게 학습할 수 있다.
\end{bulletlist}

\subsubsection*{단점}

\begin{bulletlist}
\item 사용자는 버려질 프로토타입을 평가하는 데 많은 시간을 할애할 동기가 높지 않을 수 있다.

\item 버려질 프로토타입을 버리지 않고 계속 개발하고 싶은 유혹을 받을 수 있다.
\end{bulletlist}

\subsubsection*{어려움}

\begin{bulletlist}
\item 결국 이미 실행 중인 시스템이 있기 때문에, 자신이나 고객에게 프로토타입의 필요성을 설득하기 어려울 수 있다.

\item 진화적 프로토타입을 완성하는 데 너무 많은 시간이 걸릴 수 있다. 레거시 컴포넌트에 \patref{올바른 인터페이스 제시하기}{PresentTheRightInterface}를 적용하여 프로토타입에 레거시 서비스를 위한 좋은 인터페이스를 제공하는 것을 고려하자. 
\end{bulletlist}

\subsection*{근거}

\index{브룩스, 프레드릭}
프로토타입은 특정 기술적 접근 방식이 올바른지 아닌지를 빠르게 알려줄 수 있다. \emph{맨먼스 미신(The Mythical Man-Month)}의 브룩스 \cite{Broo75a}는 처음부터 제대로 만들기는 어렵기 때문에 '버릴 것은 버려라'라고 조언한다.
\index{러브, 톰}
\index{푸트, 브라이언}
\index{요더, 조셉}
러브 \cite{Love93a}는 여기서 한 걸음 더 나아가 객체 지향 시스템의 경우 ``버릴 것을 각오하고 두 번 구현해야 한다(write two to throw away)''고 경고한다! 푸트와 요더 \cite{Foot00a}는 무엇보다도 \ind{버릴 임시 코드(Throwaway Code)}가 도메인 요구 사항을 명확히 하는 가장 좋은 방법이라고 주장하지만, 프로토타입이 ``\ind{큰 진흙 뭉치(Big Ball of Mud)}''로 발전할 위험이 있다고 경고하기도 한다.

\subsection*{관련 패턴}

\patref{뉴타운으로 가는 다리 만들기}{MakeABridgeToTheNewTown}을 적용하여 레거시 데이터를 진화하는 프로토타입으로 마이그레이션하는 것을 고려할 수 있다.

%=================================================================
%:PATTERN -- {Always Have a Running Version}
\pattern{실행 버전 항상 보유하기}{AlwaysHaveARunningVersion}


\intent{주기적으로 시스템을 리빌드하여 변경 사항에 대한 신뢰도를 높인다.}

\subsection*{문제}

올바른 길을 가고 있다고 어떻게 고객을 확신시킬 수 있는가?

\emph{이 문제는 다음과 같은 이유로 어렵다.}

\begin{bulletlist}
\item It can be hard to demo a software system under development, or to discuss problems with users since there is often no stable, running version of the system available.

\item Integrating changes from multiple versions of a system can be slow and painful.
\end{bulletlist}

\emph{그러나 이 문제를 해결할 수 있는 이유는 다음과 같다.}

\begin{bulletlist}
\item You don't have to wait until a component is ``finished'' before integrating it.
\end{bulletlist}

\subsection*{해결}

Institute a discipline of integrating new changes and developments on a daily basis.

\subsubsection*{단계}

\begin{bulletlist}
\item Have version management and configuration management systems in place.

\item Make sure you have regression tests in place for the parts you are working on.

\item Institute a discipline of short transactions for checking out system components and checking them back in again. Plan iterations to be as short as possible to allow changes to be integrated into a running system.
\end{bulletlist}

\subsection*{트레이드오프}

\subsubsection*{장점}

\begin{bulletlist}
\item You always have a working version to demo.

\item You can always have a working version to run your regression tests.

\item You can quickly validate your changes, thereby helping you to \patref{자신감 구축하기}{BuildConfidence}.
\end{bulletlist}

\subsubsection*{단점}

\begin{bulletlist}
\item You must continuously integrate changes.
\end{bulletlist}

\subsubsection*{어려움}

\begin{bulletlist}
\item Large systems may have very long build times. You may need to rearchitect the system first to enable shorter build times.

\item It can be hard to break some kinds of large modifications into meaningful updates that can be individually integrated. 
\end{bulletlist}

\subsection*{근거}

Many practitioners advocate a process of continuous integration as a way to avoid a risky and painful big-bang integration \cite{Booc94a}.

\subsection*{관련 패턴}

\patref{변경할 때마다 회귀 테스트하기}{RegressionTestAfterEveryChange} minimizes the risk of defects creeping in during integration. 

\ind{Continuous Integration} \cite{Booc94a} \cite{Beck00a} is a proven way to \patref{실행 버전 항상 보유하기}{AlwaysHaveARunningVersion}.

%=================================================================
%:PATTERN -- {Regression Test After Every Change}
\pattern{변경할 때마다 회귀 테스트하기}{RegressionTestAfterEveryChange}


\intent{Build confidence by making sure that whatever worked before still works.}

\subsection*{문제}

How can you be sure that the last change you made won't break the system?

\emph{이 문제는 다음과 같은 이유로 어렵다.}

\begin{bulletlist}
\item In a complex system, small changes can have unexpected side effects. A seemingly innocuous change may break something without this being immediately discovered.
\end{bulletlist}

\emph{그러나 이 문제를 해결할 수 있는 이유는 다음과 같다.}

\begin{bulletlist}
\item You have written test suites that express how the system should behave.
\end{bulletlist}

\subsection*{해결}

Run your regression test suite every time you think you have reached a stable state.

\subsection*{트레이드오프}

\subsubsection*{장점}

\begin{bulletlist}
\item It is easier to \patref{실행 버전 항상 보유하기}{AlwaysHaveARunningVersion}.

\item It is easier to \patref{자신감 구축하기}{BuildConfidence} as you proceed.
\end{bulletlist}

\subsubsection*{단점}

\begin{bulletlist}
\item You must relentlessly write the tests. 
\end{bulletlist}

\subsubsection*{어려움}

\begin{bulletlist}
\item The legacy system may not have adequate regression tests defined. To enable evolution, you will have to \patpgref{Grow Your Test Base Incrementally}{GrowYourTestBaseIncrementally}

\item Tests can only show that defects are present, not that they are absent. You may have failed to test precisely the aspect that you have broken.

\item Run the tests may be very time-consuming, so you might want to run only those tests that you think might be affected by your change. Categorize your tests to avoid ``ad hoc'' testing of changes, but run all the tests at least once a day.
\end{bulletlist}

\subsection*{근거}

Regression tests tell you that whatever ran before still runs. If you consistently build up tests for defects you discover and new features, you will end up with a reusable test base that gives you confidence that your changes are sound, and helps you detect problems earlier.

\index{Davis, Alan}
Davis advocates ``\patref{변경할 때마다 회귀 테스트하기}{RegressionTestAfterEveryChange}'' \cite{Davi95a} as standard Software Development practice.

\subsection*{관련 패턴}

You should have already started to \patpgref{진화 활성화를 위한 테스트 작성하기}{WriteTestsToEnableEvolution}. 

\index{Test-Driven development}A common practice in \ind{Extreme Programming} is to write tests \emph{before} you implement new functionality \cite{Jeff01a}. In the context of reengineering, you should consider writing tests that will fail before you make a change, and will pass if the change is correctly implemented. (Unfortunately it is not generally possible to design tests that will \emph{only} pass if the change is correct!)

Regression tests should help you to \patpgref{Retest Persistent Problems}{RetestPersistentProblems}.

%=================================================================
%:PATTERN -- {Make a Bridge to the New Town}
\pattern{뉴 타운으로 가는 다리 만들기}{MakeABridgeToTheNewTown}

\emph{Also Known As:}  The \ind{Bridge to the New Town} \cite{Kell00a}, \ind{Keep the Data --- Toss the Code} \cite{Brod95a}

\intent{Migrate data from a legacy system by running the new system in parallel, with a bridge in between.}

\subsection*{문제}

How do you incrementally migrate data from a legacy system to its replacement while the two systems are running in tandem?

\emph{이 문제는 다음과 같은 이유로 어렵다.}

\begin{bulletlist}
\item Some components of the legacy system are beyond repair and should be replaced.

\item Big-bang replacement of critical components is highly risky.

\item The \emph{data} manipulated by the legacy components must be kept available and alive during the migration.
\end{bulletlist}

\emph{그러나 이 문제를 해결할 수 있는 이유는 다음과 같다.}

\begin{bulletlist}
\item You have a running legacy system.
\end{bulletlist}

\subsection*{해결}

\begin{figure}
\begin{center}
\includegraphics[width=0.75\textwidth]{MigrationBridge}
\caption{A Bridge helps you to transparently transfer data to the new system.}
\figlabel{MigrationBridge}
\end{center}
\end{figure}

Make a (data) bridge that will incrementally transfer data from the legacy system to the replacement system as new components are ready to take the data over from their legacy counterparts.

\subsubsection*{단계}

\begin{bulletlist}
\item Identify legacy and replacement components that deal with the same logical data entities.

\item Implement a ``\ind{data bridge}'' which is responsible for redirecting \emph{read} requests from the new component to the legacy data source, if the data have not already been migrated. The bridge is responsible for any necessary data conversion. The new component should not be aware of the bridge.

\item Adapt the legacy component to redirect \emph{write} requests to the new component, so that the new data stay up-to-date.

\item When all the data have been transferred, remove the bridge and the legacy component.
\end{bulletlist}

\subsection*{트레이드오프}

\subsubsection*{장점}

\begin{bulletlist}
\item You can start using the new system without migrating all the legacy data.
\end{bulletlist}

\subsubsection*{단점}

\begin{bulletlist}
\item A data bridge can be tricky to implement correctly if there is not a simple mapping between the legacy data and the new data.

\item Once some of the data has been transferred, it can be hard to go back.

\item The data bridge will add a performance overhead which may or may not be acceptable.
\end{bulletlist}

\subsubsection*{어려움}

\begin{bulletlist}
\item \emph{``Stepwise migration schemes have proven very effective in large, layered business systems. They are not common in let's say CAD applications that have check in /check out persistence and a tightly coupled and very woven object net.''} \cite{Kell00a}
\end{bulletlist}

\subsection*{알려진 용도}

\index{Brodie, Michael}
\index{Stonebraker, Michael}
Brodie \& Stonebraker discuss much more thoroughly the use of data bridges and gateways in \emph{Migrating Legacy Systems} \cite{Brod95a}. 

\index{Keller, Wolfgang}
Keller in ``The Bridge to the New Town'' \cite{Kell00a} focusses more on the technical issue of migrating legacy data, and he points out numerous examples of the pattern successfully being applied.

There are many possible variants of this pattern, depending on whether the entire legacy system is to be replaced, or only a component, and whether users should be able to have access to both systems at the same time or not.

\subsection*{근거}

A bridge between the old and new systems allows you to let users start using features of the new system before it is complete. The bridge isolates the two systems from each other so that the new system can be developed according to a new architectural vision without influence from the legacy system.

\subsection*{관련 패턴}

A bridge helps you \patref{시스템 점진적 마이그레이션하기}{MigrateSystemsIncrementally} and thereby \patref{자신감 구축하기}{BuildConfidence}.

%=================================================================
%:PATTERN -- {Present the Right Interface}
\pattern{올바른 인터페이스 제시하기}{PresentTheRightInterface}

\emph{Also Known As:}  \ind{Semantic Wrapper} \cite{Ocal00a}, \ind{Sweeping it Under the Rug} \cite{Foot00a}

\intent{Wrap a legacy system to export the right abstractions, even if they are not reflected in the existing implementation.}

\subsection*{문제}

How should the new target system access legacy services during the migration process?

\emph{이 문제는 다음과 같은 이유로 어렵다.}

\begin{bulletlist}
\item The target system is not yet complete so you must rely on legacy services during the migration. 

\item The legacy system does not present the interfaces you need for the target system.

\item Implementing new components directly in terms of legacy components will bias the target towards the legacy architecture and design.
\end{bulletlist}

\emph{그러나 이 문제를 해결할 수 있는 이유는 다음과 같다.}

\begin{bulletlist}
\item You don't have to access the legacy services directly.
\end{bulletlist}

\subsection*{해결}

Identify the abstractions that you want to have in the new system, and wrap up the old software to emulate the new abstractions.

\subsubsection*{Hints}

Consider, for example, a procedural graphics library that will be used within an object-oriented system. It will be too costly and time-consuming to reimplement the library in an object-oriented way. It would be easier to wrap it as a utility class (\ie as a class with static methods but no instances), but it would be wiser to write a slightly thicker wrapper that presents a truly object-oriented interface, but is implemented using the underlying procedural abstractions. In this way the new system will not be polluted by legacy abstractions. 

\subsection*{트레이드오프}

\subsubsection*{장점}

\begin{bulletlist}
\item It is easier to wean the target system from legacy services if they can use appropriate abstractions from the start.

\item You reduce the risk that the legacy design will adversely influence the new target.
\end{bulletlist}

\subsubsection*{단점}

\begin{bulletlist}
\item The new interface may not be stable, so developers may be reluctant to use it.
\end{bulletlist}

\subsubsection*{어려움}

\begin{bulletlist}
\item It can be hard to resist the temptation to simply wrap the procedural abstractions as utility classes.
\end{bulletlist}

\subsection*{알려진 용도}

\index{O'Callaghan, Alan}
Alan O'Callaghan \cite{Ocal00a} presents this pattern as ``\ind{Semantic Wrapper}'' briefly in the context of the \ind{ADAPTOR pattern language}, which addresses migration of large-scale business-critical legacy systems to object-oriented and component-based technology. 

\subsection*{근거}

\patref{올바른 인터페이스 제시하기}{PresentTheRightInterface} frees you from thinking in terms of the legacy design and makes it easier to consider alternative approaches.

\subsection*{관련 패턴}

\patref{올바른 인터페이스 제시하기}{PresentTheRightInterface} superficially resembles an \patpgref{Adapter}{Adapter}, since both use wrappers as their implementation technique. An \patref{Adapter}{Adapter}, however, adapts an incompatible interfaces to another interface expected by its clients. \patref{올바른 인터페이스 제시하기}{PresentTheRightInterface}, on the other hand, introduces a new, more suitable interface to a legacy component. 

Be sure to \patref{폐기된 인터페이스 지원 중단하기}{DeprecateObsoleteInterfaces}. 

If the new interface implemented by the \patref{올바른 인터페이스 제시하기}{PresentTheRightInterface} is not stable, you should \patref{Distinguish Public from Published Interface}{DistinguishPublicFromPublishedInterface}.

%=================================================================
%:PATTERN -- {Distinguish Public from Published Interface}
\pattern{Distinguish Public from Published Interface}{DistinguishPublicFromPublishedInterface}

\emph{Also Known As:}  \ind{Published Interface} \cite{Ocal00a}

\intent{Facilitate parallel development by distinguishing unstable ``published interfaces'' from stable ``public interfaces''.}

\subsection*{문제}

How do you enable migration from legacy interfaces to new target interfaces while the new interfaces are still under development?

\emph{이 문제는 다음과 같은 이유로 어렵다.}

\begin{bulletlist}
\item You want to enable migration to the new target system as early as possible.

\item You do not want to freeze the interfaces of new target components too early.

\item Changing the interface to a component that is widely used will slow down development.
\end{bulletlist}

\emph{그러나 이 문제를 해결할 수 있는 이유는 다음과 같다.}

\begin{bulletlist}
\item You can control the status of the interfaces you provide.
\end{bulletlist}

\subsection*{해결}

Distinguish between public interfaces of components that are available to the rest of the system, and unstable ``published'' interfaces of components that are available within a subsystem, but are not yet ready for prime time.

\subsubsection*{Hints}

Since ``published'' interfaces are not supported by any programming language, you may have to use naming conventions, or abuse other features to achieve the desired effect.

\begin{bulletlist}
\item In \ind{Java}, consider declaring such interfaces as \lct{protected}, or giving them package scope (undeclared). When the interfaces stabilize, you may redeclare them as being \lct{public}.

\item In \ind{C++}, consider declaring components with published interfaces \lct{private} or \lct{protected}, and declare as \lct{friends} the clients that are permitted to use them. When the interfaces stabilize, redeclare the components as \lct{public}, and delete the declarations of \lct{friends}.

\item In \ind{Smalltalk}, consider declaring categories of published components. Also consider declaring published message categories to distinguish stable and unstable messages.

\item Consider decorating the names of unstable components or interfaces to indicate their ``published'' status. When the component becomes public, rename it and patch all its clients or deprecate the version with the old name (\patref{폐기된 인터페이스 지원 중단하기}{DeprecateObsoleteInterfaces}).
\end{bulletlist}

\subsection*{트레이드오프}

\subsubsection*{장점}

\begin{bulletlist}
\item Clients of published interfaces are aware that they are likely to change.
\end{bulletlist}

\subsubsection*{단점}

\begin{bulletlist}
\item Identifying an interface as ``published'' is purely a matter of convention and discipline.

\item Promoting an interface from published to public entails a certain overhead for clients who should upgrade to the new interface.
\end{bulletlist}

\subsubsection*{어려움}

\begin{bulletlist}
\item Clients can be put in a bind: should they use an unstable published interface, or continue to use the legacy service?
\end{bulletlist}

\subsection*{알려진 용도}

\ind{Published Interface} is another pattern of the \ind{ADAPTOR pattern language} \cite{Ocal00a}.

\subsection*{근거}

Clients are in a better position to evaluate the risk of using a component if they know its interface is declared to be ``published'' but not yet public.

\subsection*{관련 패턴}

When you \patref{올바른 인터페이스 제시하기}{PresentTheRightInterface} to a legacy component, the new interface may not be stable, so be careful to \patref{Distinguish Public from Published Interface}{DistinguishPublicFromPublishedInterface}. When the new interface stabilizes, or is substituted by a stable replacement component, the interface may become public.

Upgrading an interface to public may entail a change to the way it is accessed. Be sure to \patref{폐기된 인터페이스 지원 중단하기}{DeprecateObsoleteInterfaces}.

%=================================================================
%:PATTERN -- {Deprecate Obsolete Interfaces}
\pattern{폐기된 인터페이스 지원 중단하기}{DeprecateObsoleteInterfaces}

\emph{Also Known As:}  \ind{Deprecation} \cite{Stev98a}

\intent{Give clients time to react to changes to public interfaces by flagging obsolete interfaces as ``deprecated''.}

\subsection*{문제}

How do you modify an interface without invalidating all the clients?

\emph{이 문제는 다음과 같은 이유로 어렵다.}

\begin{bulletlist}
\item Changing a public interface can break many clients.

\item Leaving an obsolete interface in place will make future maintenance more difficult.

\item Not all changes are for the better.
\end{bulletlist}

\emph{그러나 이 문제를 해결할 수 있는 이유는 다음과 같다.}

\begin{bulletlist}
\item The old and the new interfaces can coexist for a period of time.
\end{bulletlist}

\subsection*{해결}

Flag the old interface as being ``deprecated'', thereby notifying clients that it will almost certainly be removed in the next upcoming release.

\subsubsection*{단계}

\begin{bulletlist}
\item You have determined that a public interface should be changed, but you do not want to break all clients. Implement the new interface, but ``deprecate'' the old one. The deprecation mechanism should inform clients that the interface has changed, and that a newer interface is recommended instead.

\item Evaluate to what extent the deprecated interface continues to be used, and whether it can be permanently retired. Consider removing it in a future release.

\item \ind{Java} supports deprecation as a language feature:
\begin{bulletlist}
\item Deprecate a feature by adding the tag \lct{@deprecated} to its javadoc documentation. The tag is not only recognized by the javadoc documentation generator, but the compiler will also generate compile-time warnings if code using deprecated features is compiled with the -deprecated option.
\end{bulletlist}

\item Other approaches are:

\begin{bulletlist}
\item Simply inform users in the documentation which interfaces are deprecated.

\item Move or rename the deprecated interface or component. Clients can continue to use them, but must adapt and recompile to continue to use the deprecated form.

\item Replace deprecated components by equivalent ones that generate run-time warnings or output warnings to a log file. 

\item Alternatively, consider configuring the programming environment or the deprecated components themselves to generate compile-time or link-time warnings.
\end{bulletlist}
\end{bulletlist}

\subsection*{트레이드오프}

\subsubsection*{장점}

\begin{bulletlist}
\item Clients do not have to immediately adapt to changes.

\item There is time to change your mind.
\end{bulletlist}

\subsubsection*{단점}

\begin{bulletlist}
\item Clients are free to ignore deprecation.
\end{bulletlist}

\subsubsection*{어려움}

\begin{bulletlist}
\item It may be hard to track down all the clients of a deprecated component.

\item It can be hard to decide when to really retire a deprecated component.

\item If you want to keep the interface but change the semantics, you may need to introduce a new component and deprecate the old one. This can be the case if certain methods should now return default values instead of throwing exceptions (or vice versa).
\end{bulletlist}

\subsection*{알려진 용도}

\index{Stevens, Perdita}
\index{Pooley, Rob}
Perdita Stevens and Rob Pooley identify \ind{Deprecation} as a common practice for managing evolving APIs in complex systems \cite{Stev98a}.

\subsection*{근거}

Deprecation gives you a window of time to evaluate the impact of a change.

%=================================================================
%:PATTERN -- {Conserve Familiarity}
\pattern{친숙도 보존하기}{ConserveFamiliarity}


\intent{Avoid radical changes that may alienate users.}

\subsection*{문제}

How do you accomplish a major overhaul of a legacy system without disrupting the way users are used to getting their job done?

\emph{이 문제는 다음과 같은 이유로 어렵다.}

\begin{bulletlist}
\item The legacy system requires significant changes.

\item The users are not happy with the legacy system, but they understand it well.
\end{bulletlist}

\emph{그러나 이 문제를 해결할 수 있는 이유는 다음과 같다.}

\begin{bulletlist}
\item You can migrate incrementally to a new solution.
\end{bulletlist}

\subsection*{해결}

Introduce only a constant, relatively low number of changes between each new release.

\subsection*{트레이드오프}

\subsubsection*{장점}

\begin{bulletlist}
\item Users do not have to change their work habits too much between releases.
\end{bulletlist}

\subsubsection*{어려움}

\begin{bulletlist}
\item Sometimes radical change is necessary. It can be hard to migrate from a command-line interface to a GUI while conserving familiarity.
\end{bulletlist}

\subsection*{근거}

Too much change between releases increases the risk of hidden defects, and decreases the chance of user acceptance. 

\index{Lehman, Manny}
\index{Belady, Les}
Lehman and Belady's ``\ind{Law of Conservation of Familiarity}'' suggests that the incremental change between releases of a system stays roughly constant over time \cite{Lehm85a}. This is a relatively natural phenomenon because to do anything else introduces unnecessary risks.

\subsection*{관련 패턴}

To \patref{친숙도 보존하기}{ConserveFamiliarity} you must \patref{시스템 점진적 마이그레이션하기}{MigrateSystemsIncrementally}. \patref{사용자 참여시키기}{InvolveTheUsers} to understand what changes will be acceptable. \patref{목표 솔루션 프로토타입하기}{PrototypeTheTargetSolution} to evaluate the potential impact of changes.

%=================================================================
%:PATTERN -- {Use Profiler Before Optimizing}
\pattern{최적화하기 전에 프로파일러 사용하기}{UseProfilerBeforeOptimizing}


\intent{Avoid squandering reengineering effort on needless ``optimizations'' by verifying where the bottlenecks are.}

\subsection*{문제}

When should you rewrite a clearly inefficient piece of code?

\emph{이 문제는 다음과 같은 이유로 어렵다.}

\begin{bulletlist}
\item When you are reengineering software, you are likely to encounter many naive algorithms in the legacy code.

\item It can be hard to predict what will impact performance, and you can lose a lot of time on pure supposition.

\item Optimized code is often more complex than simple, naive code.
\end{bulletlist}

\emph{그러나 이 문제를 해결할 수 있는 이유는 다음과 같다.}

\begin{bulletlist}
\item There are tools to tell you where you may have a performance problem.
\end{bulletlist}

\subsection*{해결}

Whenever you are tempted to optimize a ``clearly inefficient'' part of the system, first use a profiler to determine whether it is actually a bottleneck. 

Don't optimize anything unless your profiler tells you it will make a difference.

If you decide to go ahead, prepare benchmarks that will demonstrate the performance gains. 

\subsection*{트레이드오프}

\subsubsection*{장점}

\begin{bulletlist}
\item You do not waste time optimizing something that will not make a difference to overall performance.
\end{bulletlist}

\subsubsection*{단점}

\begin{bulletlist}
\item Naive algorithms will survive longer in the system.
\end{bulletlist}

\subsection*{근거}

The performance improvement that you can gain by optimizing a bit of code depends on how much time the program, spends in that code in a typical run. A profiler will tell you how much time that is.

``\ind{Do it, then do it right, then do it fast}'' is a well-known aphorism that has been credited to many different sources. Very likely its origin is outside of the field of computer science. The rationale behind it is that you risk making a system complex and hard to maintain if you become preoccupied with performance issues too early. Instead, it is better to first find a solution that works, then clean it up once you understand it. Finally, if you can identify any important performance bottlenecks, that is the time to optimize just those parts that will make a difference.

As a corollary, it may even be a good idea to replace a bit of complex, ``optimized'' code by a simpler, ``naive'' solution, if that won't severely impact performance, but will make it easier to make other changes. 

\index{Davis, Alan}
See also Davis' discussion of ``\patref{최적화하기 전에 프로파일러 사용하기}{UseProfilerBeforeOptimizing}'' \cite{Davi95a}.

\subsection*{관련 패턴}

If you \patpgref{Refactor to Understand}{RefactorToUnderstand}, you will have started the second step to ``do it right."

%=============================================================
\ifx\wholebook\relax\else
   \bibliographystyle{alpha}
   \bibliography{scg}
   \end{document}
\fi
%=============================================================
