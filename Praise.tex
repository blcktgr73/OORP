% $Author: oscar $
% $Date: 2008-03-13 16:18:43 +0100 (Thu, 13 Mar 2008) $
% $Revision: 17031 $
%=================================================================
\ifx\wholebook\relax\else
% --------------------------------------------
% Lulu:
	\documentclass[a4paper,10pt,twoside]{book}
	\usepackage[
		papersize={6.13in,9.21in},
		hmargin={.815in,.815in},
		vmargin={.98in,.98in},
		ignoreheadfoot
	]{geometry}
	\usepackage[hangul]{kotex}
	\input{common.tex}
	\pagestyle{headings}
	\setboolean{lulu}{true}
% --------------------------------------------
% A4:
%	\documentclass[a4paper,11pt,twoside]{book}
%	\input{common.tex}
%	\usepackage{a4wide}
% --------------------------------------------
	\begin{document}
	\frontmatter
	% \renewcommand{\nnbb}[2]{} % Disable editorial comments
	\sloppy
\fi
%=================================================================
~ % force vfill
\vfill
\noindent
{\large \bf 객체 지향 리엔지니어링 패턴 추천사}

\vspace{1cm}

\noindent
리팩터링하는 ``방법''은 이미 여러 문헌에서 잘 다루고 있다. 하지만 ``언제''와 ``왜''는 경험을 통해서만 배울 수 있다. 이 책은 시스템 재설계를 언제 시작해야 하는지, 언제 중단해야 하는지, 그리고 리팩터링을 통해 어떤 효과를 기대할 수 있는지를 배우는 데 도움이 될 것이다.

\hfill--- \emph{켄트 벡, Three Rivers Institute 디렉터}\\[0.2cm]

\noindent
이 책은 실용적이고 실무적인 리엔지니어링 지식과 전문성을 이해하기 쉽고 사용하기 쉬운 형태로 제시하고 있다. 따라서 이 책의 패턴은 리엔지니어링(reengineering)을 사용하는 데 관심이 있는 모든 사람이 자신의 작업에 적용하는 데 도움이 된다. 이 책이 진작에 내 서재에 있었더라면 바램이 있다.

\hfill--- \emph{프랭크 부시만, 지멘스 시니어 수석 엔지니어}\\[0.2cm]

\noindent
이 책은 제목이 이야기하는 것 그 이상이다. 효과적인 리엔지니어링이란 다른 사람의 코드를 의도적이고 효율적으로 읽어 예측 가능한 변화를 만들어내는 것이다. 저자가 숙련된 리엔지니어링 행동의 패턴으로 강조하는 것과 동일한 프로세스는 읽기 쉽고 유지 관리가 가능한 소프트웨어 시스템을 만드는 데 필요한 기술로 쉽게 적용할 수 있다.

\hfill--- \emph{아델 골드버그, Neometron, Inc.}\\[0.2cm]

\noindent
만약 어떤 사람이 내 사무실로 많은 문서와 CD 두 장이 들어 있는 커다란 상자(회사에서 리엔지니어링하고자 하는 소프트웨어의 설치 디스크와 자료)를 가져온다면, 이 책의 저자를 제 곁에 두게 되어 기쁠 것 같다. 그렇지 않다면 저자들의 책을 갖는 것이 차선책이다. 이 책에는 거창한 내용도, 과대 광고도, 쉽다는 약속도 없습니다. 그저 프로젝트를 해결하는 데 유용한 지침이 담긴 현실적이고 읽기 쉬우며 매우 유용한 책입니다. 그 고객이 사무실에 도착하기 전에 이 책을 구입하여 살펴보자! 여러분과 여러분의 회사가 많은 고민을 덜 수 있을 것입다.

\hfill--- \emph{린다 라이징, 인디펜던트 컨설턴트}

\vfill

%=============================================================
\ifx\wholebook\relax\else
   \bibliographystyle{alpha}
   \bibliography{scg}
   \end{document}
\fi
%=============================================================
