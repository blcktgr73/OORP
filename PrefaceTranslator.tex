% $Author: oscar $
% $Date: 2009-09-15 16:53:48 +0200 (Tue, 15 Sep 2009) $
% $Revision: 29111 $
%=================================================================
\ifx\wholebook\relax\else
% --------------------------------------------
% Lulu:
	\documentclass[a4paper,10pt,twoside]{book}
	\usepackage[
		papersize={6.13in,9.21in},
		hmargin={.815in,.815in},
		vmargin={.98in,.98in},
		ignoreheadfoot
	]{geometry}
	\usepackage[hangul]{kotex}
	\input{common.tex}
	\pagestyle{headings}
	\setboolean{lulu}{true}
% --------------------------------------------
% A4:
%	\documentclass[a4paper,11pt,twoside]{book}
%	\input{common.tex}
%	\usepackage{a4wide}
% --------------------------------------------
	\begin{document}
	\frontmatter
	% \renewcommand{\nnbb}[2]{} % Disable editorial comments
	\sloppy
\fi

%=================================================================
\chapter{역자 서문}
\chalabel{PrefaceTranslator}

%=================================================================

\section*{이 책과의 인연}

이 책은 애자일 코칭의 선구자이신 김창준님께서 제가 관심을 가질 만한 책이라고 소개해 주셔서 알게 되었다.
『객체 지향 리엔지니어링 패턴(Object-Oriented Reengineering Patterns)』은 이미 출판된 적이 있지만, 크리에이티브 커먼즈 라이선스로 공개되어 있다.
번역을 시작하기 전에 저자인 세르게이 디마이어(Serge Demeyer) 교수에게 직접 연락하여 번역 허락을 받았고, 2024년 6월 20일부터 본격적인 번역 작업을 시작하였다.

오랜만에 LaTeX을 사용하며 학위 논문을 작성하던 시절이 떠올랐다.
번역 과정에서 인상 깊었던 내용들은 블로그에도 함께 정리하였다.

\section*{이 책의 구성}

이 책은 레거시 소프트웨어 시스템을 효과적으로 이해하고 개선하는 방법을 패턴 형식으로 제시한다.
총 10개의 장으로 구성되어 있으며, 크게 세 부분으로 나눌 수 있다.

\textbf{1부 개요(1장)}에서는 소프트웨어 리엔지니어링의 기본 개념을 소개한다.
레거시 시스템이 비즈니스에 중요함에도 불구하고 많은 비용이 들어 업그레이드나 교체가 어려운 이유,
리엔지니어링을 통해 시스템의 복잡성을 줄이고 적절한 비용으로 시스템을 계속 사용하고 적응시킬 수 있는 방법을 다룬다.
리버스 엔지니어링과 리엔지니어링의 차이점을 명확히 하고, 리엔지니어링 패턴의 양식을 제시한다.

\textbf{2부 리버스 엔지니어링(2--5장)}에서는 처음 접하는 레거시 시스템을 체계적으로 이해하는 방법을 설명한다.
2장 ``방향 설정''에서는 리엔지니어링 프로젝트의 목표와 범위를 정의하는 방법을,
3장 ``첫 번째 접근''에서는 낯선 시스템에 첫발을 내딛는 전략을 제시한다.
4장 ``초기 이해''에서는 시스템의 데이터와 구조를 파악하는 기법을,
5장 ``상세 모델 포착''에서는 실제 코드를 읽고 이해하여 더욱 정밀한 모델을 구축하는 방법을 다룬다.

\textbf{3부 리엔지니어링(6--10장)}에서는 시스템을 실제로 개선하는 구체적인 전략과 패턴을 제공한다.
6장 ``테스트: 당신의 생명 보험''에서는 간과하기 쉽지만 리엔지니어링 과정에서 필수적인 테스트 전략을 설명한다.
7장 ``마이그레이션 전략''에서는 시스템을 새로운 플랫폼이나 아키텍처로 옮기는 방법을,
8장 ``중복 코드 감지''에서는 코드 중복을 찾아내고 제거하는 기법을,
9장 ``책임 재배치''에서는 클래스와 메서드의 책임을 더 나은 구조로 재구성하는 방법을,
10장 ``조건문을 다형성으로 변환''에서는 복잡한 조건문을 객체 지향적인 다형성으로 개선하는 패턴을 제시한다.

\section*{이 책이 나아가길 바라는 방향}

이 책의 저자들은 수많은 프로젝트에서 쌓은 풍부한 경험을 바탕으로,
완벽하지 않은 소프트웨어도 지속적인 개선을 통해 더 나아질 수 있다는 것을 패턴으로 보여준다.
처음 설계가 부족했던 시스템이라도, 체계적인 리엔지니어링을 거치면
유지보수가 용이하고 확장 가능한 시스템으로 진화할 수 있다.
이것이야말로 이 책이 전하는 가장 중요한 메시지이다.

이 번역이 기존 프로젝트를 분석하고 더 나은 구조로 개선하려는 개발자들에게 작은 도움이 되길 바란다.
레거시 코드는 단순히 낡은 코드가 아니라, 비즈니스 가치를 담고 있는 자산이다.
이 책에 담긴 패턴과 전략들이 그 자산을 더욱 가치 있게 만드는 여정에 동반자가 되기를 희망한다.

\vspace{1cm}
\noindent
2024년\\
역자 이승범

%=============================================================
\ifx\wholebook\relax\else
   \bibliographystyle{alpha}
   \bibliography{scg}
   \end{document}
\fi
%=============================================================

